\documentclass[12pt,a4paper]{article}
\usepackage[T1]{fontenc}
\usepackage[utf8]{vietnam}
\usepackage{amsmath}
\usepackage{amssymb}
\usepackage{amsfonts}
\usepackage{enumitem}
\usepackage{fancyhdr}
\usepackage{changepage}
\usepackage{lipsum}
\usepackage{tikz}
\usepackage{stackrel}
\usepackage{amsthm}
\usepackage{graphicx}
\usepackage{listings}
\usepackage{color}
\usepackage{tabularx}
\usepackage{transparent}
\usepackage{amsmath}
\usepackage{pgfplots}
\usepackage{filecontents}
\usepackage{caption}
\usepackage{listings}
\usepackage{xcolor}
\usepackage{filecontents}
\definecolor{codegreen}{rgb}{0,0.6,0}
\definecolor{codegray}{rgb}{0.5,0.5,0.5}
\definecolor{codepurple}{rgb}{0.58,0,0.82}
\definecolor{backcolour}{rgb}{0.95,0.95,0.92}
\lstdefinestyle{mystyle}{
    backgroundcolor=\color{backcolour},   
    commentstyle=\color{codegreen},
    keywordstyle=\color{magenta},
    numberstyle=\tiny\color{codegray},
    stringstyle=\color{codepurple},
    basicstyle=\footnotesize,
    breakatwhitespace=false,         
    breaklines=true,                 
    captionpos=b,                    
    keepspaces=true,                 
    numbers=left,                    
    numbersep=5pt,                  
    showspaces=false,                
    showstringspaces=false,
    showtabs=false,                  
    tabsize=2
}
\lstset{ 
  language=Python, 
  basicstyle=\ttfamily\footnotesize,
  keywordstyle=\color{blue},
  stringstyle=\color{red},
  commentstyle=\color{gray},
  morecomment=[l][\color{magenta}]{\#},
  numbers=left, 
  numberstyle=\tiny\color{gray},
  stepnumber=1,
  numbersep=5pt,
  frame=single,
  breaklines=true
}
\usepackage[margin=1.2cm]{geometry}
\title{\LARGE{\textbf{Nhập môn phân tích độ phức tạp thuật toán-21TN}}}
\author{TRẦN MINH HOÀNG-21120075}
\date{} % Ngày không cần hiển thị
\begin{document}
\maketitle
\title{Phân tích độ phức tạp của đoạn mã}
\author{}
\date{}
\maketitle

\section*{Đề bài}

Cho đoạn mã bên dưới:

\begin{lstlisting}[language=C++]
int process(int n){
    int i = n, m = 1, res = 0;
    while (i > 0){
        int j = 1;
        while (j < m){
            res = i * j;
            j = j + ?;
        }
        m = m + 1;
        i = i - @;
    }
    return res;
}
\end{lstlisting}

Hãy thay mỗi ký tự \texttt{?}, \texttt{@} ở dòng code 7, 10 lần lượt bởi các cách sau: $(1,1)$, $(j,j)$, $(1,j)$, $(j,1)$. Với mỗi cách, thực hiện đếm số phép gán và ước lượng độ phức tạp tương ứng.

\section*{Phân tích chi tiết}

\subsection*{Trường hợp 1: (1, 1)}
\begin{lstlisting}[language=C++]
int process(int n){
    int i = n, m = 1, res = 0;
    while (i > 0){
        int j = 1;
        while (j < m){
            res = i * j;
            j = j + 1;
        }
        m = m + 1;
        i = i - 1;
    }
    return res;
}
\end{lstlisting}

\textbf{Phân tích số phép gán và độ phức tạp:}
\begin{itemize}
    \item Vòng lặp ngoài (while \texttt{i > 0}) chạy $n$ lần vì $i$ giảm từ $n$ đến $0$ với bước giảm là $1$.
    \item Vòng lặp trong (while \texttt{j < m}) chạy số lần phụ thuộc vào giá trị của $m$. Cụ thể:
    \begin{itemize}[label=$\bullet$]
        
        \item Đến lần thứ $k$, vòng lặp trong chạy $k-1$ lần.
    \end{itemize}
    \item Tổng số lần vòng lặp trong chạy là: 
    \[
    \sum_{k=1}^{n-1} k = \frac{(n-1)n}{2}
    \]
    \item Số phép gán \texttt{j = j + 1} trong vòng lặp trong: 
    \[
    \frac{(n-1)n}{2}
    \]
    \item Số phép gán \texttt{m = m + 1} thực hiện $n$ lần.
    \item Số phép gán \texttt{i = i - 1} thực hiện $n$ lần.
    \item Tổng số phép gán là:
    \[
    \frac{(n-1)n}{2} + 2n = O(n^2)
    \]
\end{itemize}

\subsection*{Trường hợp 2: (j, j)}
\begin{lstlisting}[language=C++]
int process(int n){
    int i = n, m = 1, res = 0;
    while (i > 0){
        int j = 1;
        while (j < m){
            res = i * j;
            j = j + j;
        }
        m = m + 1;
        i = i - j;
    }
    return res;
}
\end{lstlisting}

\textbf{Phân tích số phép gán và độ phức tạp:}
\begin{itemize}
    \item Vòng lặp ngoài (while \texttt{i > 0}) chạy ít hơn $n$ lần do $i$ giảm nhanh.
    \item Vòng lặp trong (while \texttt{j < m}) chạy log($m$) lần, vì $j$ tăng gấp đôi mỗi lần.
    \item Số lần thực hiện phép gán \texttt{j = j + j} trong vòng lặp trong là log($m$).
    \item Số lần thực hiện phép gán \texttt{m = m + 1} là $n$ lần.
    \item Số lần thực hiện phép gán \texttt{i = i - j} ít hơn $n$ lần, ước lượng là $O(n)$.
    \item Độ phức tạp tổng quát là $O(n \log(n))$.
\end{itemize}

\subsection*{Trường hợp 3: (1, j)}
\begin{lstlisting}[language=C++]
int process(int n){
    int i = n, m = 1, res = 0;
    while (i > 0){
        int j = 1;
        while (j < m){
            res = i * j;
            j = j + 1;
        }
        m = m + 1;
        i = i - j;
    }
    return res;
}
\end{lstlisting}

\textbf{Phân tích số phép gán và độ phức tạp:}
\begin{itemize}
    \item Vòng lặp ngoài (while \texttt{i > 0}) chạy ít hơn $n$ lần do $i$ giảm nhanh.
    \item Vòng lặp trong (while \texttt{j < m}) chạy $m-1$ lần, vì $j$ tăng từ 1 lên $m-1$.
    \item Tổng số lần vòng lặp trong chạy là:
    \[
    \sum_{k=1}^{n-1} k = \frac{(n-1)n}{2}
    \]
    \item Số lần thực hiện phép gán \texttt{j = j + 1} trong vòng lặp trong là:
    \[
    \frac{(n-1)n}{2}
    \]
    \item Số lần thực hiện phép gán \texttt{m = m + 1} là $n$ lần.
    \item Số lần thực hiện phép gán \texttt{i = i - j} ít hơn $n$ lần, ước lượng là $O(n)$.
    \item Độ phức tạp tổng quát là $O(n \log(n))$.
\end{itemize}

\subsection*{Trường hợp 4: (j, 1)}
\begin{lstlisting}[language=C++]
int process(int n){
    int i = n, m = 1, res = 0;
    while (i > 0){
        int j = 1;
        while (j < m){
            res = i * j;
            j = j + j;
        }
        m = m + 1;
        i = i - 1;
    }
    return res;
}
\end{lstlisting}

\textbf{Phân tích số phép gán và độ phức tạp:}
\begin{itemize}
    \item Vòng lặp ngoài (while \texttt{i > 0}) chạy $n$ lần vì $i$ giảm từ $n$ đến $0$ với bước giảm là $1$.
    \item Vòng lặp trong (while \texttt{j < m}) chạy log($m$) lần do $j$ tăng gấp đôi mỗi lần.
    \item Số lần thực hiện phép gán \texttt{j = j + j} trong vòng lặp trong là log($m$).
    \item Số lần thực hiện phép gán \texttt{m = m + 1} là $n$ lần.
    \item Số lần thực hiện phép gán \texttt{i = i - 1} là $n$ lần.
    \item Độ phức tạp tổng quát là $O(n \log(n))$.
\end{itemize}

\section*{Tóm tắt}
\begin{itemize}
    \item Trường hợp $(1,1)$: $O(n^2)$
    \item Trường hợp $(j,j)$: $O(n \log(n))$
    \item Trường hợp $(1,j)$: $O(n \log(n))$
    \item Trường hợp $(j,1)$: $O(n \log(n))$
\end{itemize}

Như vậy, các trường hợp $(j,j)$, $(1,j)$, và $(j,1)$ đều có độ phức tạp thấp hơn $O(n \log(n))$ so với $O(n^2)$ của trường hợp $(1,1)$.

\end{document}
