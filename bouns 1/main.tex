\documentclass[12pt,a4paper]{article}
\usepackage[T1]{fontenc}
\usepackage[utf8]{vietnam}
\usepackage{amsmath}
\usepackage{amssymb}
\usepackage{amsfonts}
\usepackage{enumitem}
\usepackage{fancyhdr}
\usepackage{changepage}
\usepackage{lipsum}
\usepackage{tikz}
\usepackage{stackrel}
\usepackage{amsthm}
\usepackage{graphicx}
\usepackage{listings}
\usepackage{color}
\usepackage{tabularx}
\usepackage{transparent}
\usepackage{amsmath}
\usepackage{pgfplots}
\usepackage{filecontents}
\usepackage{caption}
\usepackage{listings}
\usepackage{xcolor}
\usepackage{filecontents}
\definecolor{codegreen}{rgb}{0,0.6,0}
\definecolor{codegray}{rgb}{0.5,0.5,0.5}
\definecolor{codepurple}{rgb}{0.58,0,0.82}
\definecolor{backcolour}{rgb}{0.95,0.95,0.92}
\lstdefinestyle{mystyle}{
    backgroundcolor=\color{backcolour},   
    commentstyle=\color{codegreen},
    keywordstyle=\color{magenta},
    numberstyle=\tiny\color{codegray},
    stringstyle=\color{codepurple},
    basicstyle=\footnotesize,
    breakatwhitespace=false,         
    breaklines=true,                 
    captionpos=b,                    
    keepspaces=true,                 
    numbers=left,                    
    numbersep=5pt,                  
    showspaces=false,                
    showstringspaces=false,
    showtabs=false,                  
    tabsize=2
}
\lstset{ 
  language=Python, 
  basicstyle=\ttfamily\footnotesize,
  keywordstyle=\color{blue},
  stringstyle=\color{red},
  commentstyle=\color{gray},
  morecomment=[l][\color{magenta}]{\#},
  numbers=left, 
  numberstyle=\tiny\color{gray},
  stepnumber=1,
  numbersep=5pt,
  frame=single,
  breaklines=true
}
\usepackage[margin=1.2cm]{geometry}
\title{\LARGE{\textbf{Nhập môn phân tích độ phức tạp thuật toán-21TN}}}
\author{TRẦN MINH HOÀNG-21120075}
\date{} % Ngày không cần hiển thị
\begin{document}
\maketitle
\begin{center}
    \textbf{BÀI TẬP ĐIỂM CỘNG LẦN 2}
\end{center}
\author{}
\date{}
\maketitle

\section*{Đề bài}

Cho đoạn mã bên dưới:

\begin{lstlisting}[language=C++]
int process(int n){
    int i = n, m = 1, res = 0;
    while (i > 0){
        int j = 1;
        while (j < m){
            res = i * j;
            j = j + ?;
        }
        m = m + 1;
        i = i - @;
    }
    return res;
}
\end{lstlisting}

Hãy thay mỗi ký tự \texttt{?}, \texttt{@} ở dòng code 7, 10 lần lượt bởi các cách sau: $(1,1)$, $(j,j)$, $(1,j)$, $(j,1)$. Với mỗi cách, thực hiện đếm số phép gán và ước lượng độ phức tạp tương ứng.

\section*{Phân tích chi tiết}

\subsection*{Trường hợp 1: (1, 1)}
\begin{lstlisting}[language=C++]
int process(int n){
    int i = n, m = 1, res = 0;
    while (i > 0){
        int j = 1;
        while (j < m){
            res = i * j;
            j = j + 1;
        }
        m = m + 1;
        i = i - 1;
    }
    return res;
}
\end{lstlisting}
Gọi $T(n) $ là số phép gán:
\[
    T(n)=3+\sum_{i = 1}^{n}(3+\sum_{j=1}^{j<m}2)
\]
\textbf{Phân tích số phép gán và độ phức tạp:}

\begin{align*}
    T(n)&=3+\sum_{i = 1}^{n}(3+\sum_{j=1}^{j<m}2)\\
    &=3+\sum_{i = 1}^{n}(3+2m)\\
\end{align*}
\begin{itemize}
    \item Qua mỗi vòng lặp i giảm 1 đơn vị, m tăng 1 đơn vị $\Rightarrow m=n+1-i$
\end{itemize}
\begin{align*}
    T(n)&=3+\sum_{i = 1}^{n}(3+2n-2i)\\
    &=3+5n+2n^2-2\sum_{i=1}^{n}i\\
    &=3+5n+2n^2-n(n+1)\\
    &=n^2+4n+3 \in \theta(n^2)
\end{align*}
\subsection*{Trường hợp 2: (j, j)}
\begin{lstlisting}[language=C++]
int process(int n){
    int i = n, m = 1, res = 0;
    while (i > 0){
        int j = 1;
        while (j < m){
            res = i * j;
            j = j + j;
        }
        m = m + 1;
        i = i - j;
    }
    return res;
}
\end{lstlisting}

\textbf{Phân tích số phép gán và độ phức tạp:}
\begin{itemize}
    \item j được tăng lớn hơn hoặc bằng m
    \item i giảm một khoảng là m hay nói cách khác i=i-m
\end{itemize}
\begin{align*}
    T(n)&=3+\sum_{i = 1}^{n}(3+\sum_{j=1,j*=2}^{j<m}2)\\
    &=3+\sum_{i=n,i-=m}{i>0}(3+2\log_{2}(m))
\end{align*}
\begin{itemize}
    \item Gọi $a= m_{max}$ thỏa $a(a+1)=2n \Rightarrow a \thickapprox \sqrt{2n}$
\end{itemize}
\begin{align*}
    T(n)&=3+3a+2(\log_{2}(1)+\log_{2}(2)+\cdots+\log_{2}(a))\\
    &\in theta(a\log a)=\theta(\sqrt{2n}\log\sqrt{2n})=\theta(\sqrt{n}\log n)\\
\end{align*}
\subsection*{Trường hợp 3: (1, j)}
\begin{lstlisting}[language=C++]
int process(int n){
    int i = n, m = 1, res = 0;
    while (i > 0){
        int j = 1;
        while (j < m){
            res = i * j;
            j = j + 1;
        }
        m = m + 1;
        i = i - j;
    }
    return res;
}
\end{lstlisting}

\begin{align*}
    T(n)&=3+\sum_{i=n,i-=m}{i>0}(3+2\log_{2}(m))\\
    &=3+\sum_{i=n,i-=m}{i>0}(3+2m)\\
    &=3+3a+a(a+1)\\
    &=3+3\sqrt{2n}+2n \in \theta(n).
\end{align*}

\subsection*{Trường hợp 4: (j, 1)}
\begin{lstlisting}[language=C++]
int process(int n){
    int i = n, m = 1, res = 0;
    while (i > 0){
        int j = 1;
        while (j < m){
            res = i * j;
            j = j + j;
        }
        m = m + 1;
        i = i - 1;
    }
    return res;
}
\end{lstlisting}
\begin{align*}
    T(n)&=3+\sum_{i=1}{n}(3+\sum_{j=1,j*=2}{m}2)\\
    &=3+\sum_{i=1}{n}(3+2\log_{2}m)\\
    &=3+3n+2(\log_{2}(1)+\log_{2}(2)+\cdots+\log_{2}(n))\\
    &\in \theta(n\log n)
\end{align*}
\section*{Tóm tắt}
\begin{itemize}
    \item Trường hợp $(1,1)$: $O(n^2)$
    \item Trường hợp $(j,j)$: $O(\sqrt{n} \log n)$
    \item Trường hợp $(1,j)$: $O(n)$
    \item Trường hợp $(j,1)$: $O(n \log n)$
\end{itemize}
\end{document}
