\documentclass[12pt,a4paper]{article}
\usepackage[T1]{fontenc}
\usepackage[utf8]{vietnam}
\usepackage{amsmath}
\usepackage{amssymb}
\usepackage{amsfonts}
\usepackage{enumitem}
\usepackage{fancyhdr}
\usepackage{changepage}
\usepackage{lipsum}
\usepackage{tikz}
\usepackage{stackrel}
\usepackage{amsthm}
\usepackage{graphicx}
\usepackage{listings}
\usepackage{color}
\usepackage{tabularx}
\usepackage{transparent}
\usepackage{amsmath}
\usepackage{pgfplots}
\usepackage{filecontents}
\usepackage{caption}
\usepackage{listings}
\usepackage{xcolor}
\usepackage{filecontents}
\definecolor{codegreen}{rgb}{0,0.6,0}
\definecolor{codegray}{rgb}{0.5,0.5,0.5}
\definecolor{codepurple}{rgb}{0.58,0,0.82}
\definecolor{backcolour}{rgb}{0.95,0.95,0.92}
\lstdefinestyle{mystyle}{
    backgroundcolor=\color{backcolour},   
    commentstyle=\color{codegreen},
    keywordstyle=\color{magenta},
    numberstyle=\tiny\color{codegray},
    stringstyle=\color{codepurple},
    basicstyle=\footnotesize,
    breakatwhitespace=false,         
    breaklines=true,                 
    captionpos=b,                    
    keepspaces=true,                 
    numbers=left,                    
    numbersep=5pt,                  
    showspaces=false,                
    showstringspaces=false,
    showtabs=false,                  
    tabsize=2
}
\lstset{ 
  language=Python, 
  basicstyle=\ttfamily\footnotesize,
  keywordstyle=\color{blue},
  stringstyle=\color{red},
  commentstyle=\color{gray},
  morecomment=[l][\color{magenta}]{\#},
  numbers=left, 
  numberstyle=\tiny\color{gray},
  stepnumber=1,
  numbersep=5pt,
  frame=single,
  breaklines=true
}
\usepackage[margin=1.2cm]{geometry}
\title{\LARGE{\textbf{Nhập môn phân tích độ phức tạp thuật toán-21TN}}}
\author{TRẦN MINH HOÀNG-21120075}
\date{} % Ngày không cần hiển thị
\begin{document}
\maketitle
\begin{center}
    \textbf{BÀI TẬP ĐIỂM CỘNG LẦN 2}
\end{center}
\section*{Thuật toán đếm số cặp trội hơn}

Cho danh sách \( \ell \) gồm điểm thi hai môn \( X, Y \) của \( n \) sinh viên (SV), trong đó SV thứ \( i \) với \( 1 \leq i \leq n \) sẽ có một cặp điểm \( (x_i, y_i) \) là các số thực thuộc \([0; 10]\); để đơn giản, ta xét tất cả giá trị đều phân biệt. Yêu cầu đặt ra là cần đếm số cặp \((SV_i, SV_j)\) mà \( SV_j \) có điểm trội hơn so với \( SV_i \), tức là \( x_j > x_i \) và \( y_j > y_i \). Ta muốn xây dựng thuật toán divide-and-conquer để giải quyết bài toán này với ý tưởng sơ lược như sau:

\begin{itemize}
    \item \textbf{Bước divide}: sắp xếp \( \ell \) theo điểm của môn \( X \) rồi chia thành hai nửa bằng nhau là \( \ell_{\text{left}}, \ell_{\text{right}} \).
    \item \textbf{Bước conquer}: tiếp theo, ta đếm số cặp trội hơn giữa \( \ell_{\text{left}} \) và \( \ell_{\text{right}} \) bằng cách sắp xếp mỗi danh sách con theo môn \( Y \). Xét hai SV có điểm môn \( Y \) thấp nhất ở cả hai danh sách rồi thực hiện so sánh và đếm, mỗi bước sẽ loại đi được một trong hai SV, cứ tiếp tục như thế cho đến hết.
\end{itemize}

a) Hãy mô tả cụ thể hơn các ý tưởng ở trên để xây dựng chi tiết thuật toán divide-and-conquer để giải quyết bài này. Từ đó, thiết lập công thức đệ quy để ước lượng số phép so sánh.

\textbf{Mô tả thuật toán chi tiết:}

\begin{enumerate}
    \item \textbf{Sắp xếp danh sách} \( \ell \) theo điểm của môn \( X \):
    \begin{itemize}
        \item Giả sử ta sắp xếp danh sách \( \ell \) theo thứ tự tăng dần của điểm \( X \).
        \item Chia danh sách \( \ell \) thành hai phần bằng nhau \( \ell_{\text{left}} \) và \( \ell_{\text{right}} \).
    \end{itemize}
    \item \textbf{Đếm số cặp trội hơn giữa \( \ell_{\text{left}} \) và \( \ell_{\text{right}} \)}:
    \begin{itemize}
        \item Ta cần sắp xếp \( \ell_{\text{left}} \) và \( \ell_{\text{right}} \) theo điểm môn \( Y \).
        \item Sử dụng merge process để đếm số cặp \((SV_i, SV_j)\) mà \( SV_j \) có điểm trội hơn so với \( SV_i \) (tức là \( x_j > x_i \) và \( y_j > y_i \)).
        \item Giả sử \( \ell_{\text{left}} \) có \( m \) phần tử và \( \ell_{\text{right}} \) có \( n \) phần tử. Để tìm các cặp trội hơn, ta sẽ duyệt từng phần tử của \( \ell_{\text{left}} \) và \( \ell_{\text{right}} \), kiểm tra điều kiện \( y_j > y_i \) để đếm số cặp trội hơn.
    \end{itemize}
    \item \textbf{Kết hợp kết quả}:
    \begin{itemize}
        \item Đếm số cặp trội hơn trong \( \ell_{\text{left}} \).
        \item Đếm số cặp trội hơn trong \( \ell_{\text{right}} \).
        \item Đếm số cặp trội hơn giữa \( \ell_{\text{left}} \) và \( \ell_{\text{right}} \).
    \end{itemize}
\end{enumerate}

\textbf{Công thức đệ quy:}

\begin{itemize}
    \item Gọi \( T(n) \) là số phép so sánh cần thiết để đếm số cặp trội hơn trong danh sách có \( n \) sinh viên.
    \item Khi chia danh sách \( n \) phần tử thành hai phần \( n/2 \), ta cần sắp xếp từng phần và đếm số cặp trội hơn giữa hai phần.
    \item Công thức đệ quy sẽ là:
    \[
    T(n) = 2T\left(\frac{n}{2}\right) + O(n)
    \]
    \( O(n) \) là độ phức tạp của việc đếm số cặp trội hơn giữa \( \ell_{\text{left}} \) và \( \ell_{\text{right}} \).
\end{itemize}

b) Bằng cách thích hợp, ước lượng độ phức tạp của thuật toán theo hệ thức tìm được ở a).

Sử dụng công thức đệ quy từ phần a:

\[
T(n) = 2T\left(\frac{n}{2}\right) + O(n)
\]

Đây là công thức đệ quy dạng điển hình cho thuật toán "divide-and-conquer", giống với công thức của Merge Sort. Ta có thể giải công thức này bằng phương pháp cây đệ quy hoặc định lý Master.

Sử dụng định lý Master cho \( T(n) = 2T(n/2) + n \):

\begin{itemize}
    \item \( a = 2 \), \( b = 2 \), \( f(n) = n \).
    \item Ta có \( n^{\log_b a} = n^{\log_2 2} = n \).
\end{itemize}

Theo định lý Master, nếu \( f(n) = O(n^{\log_b a}) \), thì \( T(n) = O(n^{\log_b a} \log n) = O(n \log n) \).

Vậy, độ phức tạp của thuật toán là \( O(n \log n) \).
\end{document}
